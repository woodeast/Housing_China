% Wealth inequality and housing price
\documentclass[a4paper,10pt]{article}
\usepackage[left=2cm,top=2cm,bottom=2cm,right=3cm,head=0.5cm,foot=0.5cm]{geometry}
\usepackage{graphicx}
\usepackage{hyperref}
\usepackage{amssymb,amsmath}
\usepackage{enumerate}
\usepackage{chngpage}

%\sloppy
%\definecolor{lightgray}{gray}{0.5}
%\setlength{\parindent}{0pt}
%\title{Wealth inequality and housing price}
%\author{F\'{u}d\`{o}ng Zh\={a}ng}


\newif\iffudongxingtongonly
%\fudongxingtongonlytrue
\fudongxingtongonlyfalse


\begin{document}


\begin{center}
\Large{Wealth inequality and discrete housing choice}\\
\mskip
\normalsize Xintong Yang and Fudong Zhang\let\thefootnote\relax\footnotemark*
\let\thefootnote\relax\footnotetext{*Respectively: University of California, Santa Barbara (e-mail: \href{mailto:xintongyang@umail.ucsb.edu}{\tt xintongyang@umail.ucsb.edu}), University of Michigan (e-mail: \href{mailto:fudongzh@umich.edu}{\tt fudongzh@umich.edu}).}\\
\mskip
\normalsize \today\\
\end{center}

\bigskip

\subsection*{Model}

The economy is a standard overlapping generations model, extended to include the existence of housing assets, collateralized borrowing and heterogeneous endowments.

\paragraph{Household preferences and endowments.}
Households receive utility from consumption $c$ and service flows from housing $s$ rented from the rental market. Note that $s$ does not tie up with the housing asset owned by a household, i.e., we allow for $s\ne h$, where $h$ is the household's housing asset position.

To be specific, we assume the following momentary utility function

\[
U(c,s) = \frac{(c^\alpha s^{1-\alpha})^{1-\gamma}}{1-\gamma}.
\]
In our benchmark model, we start with a log-utility case ($\gamma=1$).

The agent in the economy is born with initial wealth $b_0\in B_0$ and initial housing endowment $h_0 \in H_0$. Each agent lives for $N$ periods with certainty, receives an income stream throughout her lifetime $\{y_n\}_{n=1}^{N}\in Y$.
%\footnote{We allow growth in the aggregate housing if $h_0>0$}.
The period by period discount factor is constant at $\beta$ across all the agents.

Finally, the same mass of new agents are born at the beginning of each period, with the same distribution of initial characteristics over-time.

\paragraph{Markets and timing.} There are two asset markets in the economy, a bond market and a housing asset market. 

At the beginning of each period $t$, agents can trade one-period fully enforceable bond $b_t$ which pays a gross interest rate of $1+r_t$ one period later. Negative amounts of this bond denote a debt position. Households cannot borrow more than a fraction $M_h$ of their housing asset or a fraction $M_y$ of their expected future earnings.

At the same time, households can trade housing $h$ in the asset market, but only at a quantity larger or equal to $\underline{h}$ with unit price $P(t)$ (the endowment income in the same period is the numeraire).\footnote{If an agent enters her end-life period with $0<h<\underbar{h}$, her housing asset at the end of the period will be randomly assigned to a newly born agent the next period as her endowment.} Each unit of housing asset generates a unit of flow housing service each period. There is no depreciation for housing asset and the total housing stock in the economy is fixed at $H_t=\bar{H}$.

Right after asset markets close, households participate in the goods market and housing rental market to purchase the current period consumptions. The goods market is standard. Note that for simplification, unlike the housing asset market, we allow the trading unit of $s$ to be continuous. Depending on one's housing asset holding position, an agent either buy or sell a certain amount of housing services $s$ for the current period, at market unit price $R(t)$.

\paragraph{The household's problem.}
%$\underline{a},\underbar{a}$
It is convenient to write the household's optimization problem recursively. At period $t$, the state variables are $b$ and $h$. The households observe current prices: $r_t, R(t), P(t)$. The dynamic problem of an age $n<N$ household is

%\[
%V_n(b,h)=\max_{I^{h}\in\{0,1\}} \{I^{h}V_n^{a}(b,h),(1-I^{h})V_n^{wa}(b,h)\},
%\]
%where $I^{h}=1$ denotes that the agent makes adjustment on her housing asset and $I^{h}=0$ denotes that without adjustment. $V_n^{a}(b,h)$ and $V_n^{wa}(b,h)$ are corresponding value functions, given by

\[
V_n(b,h)=\max \{V_n^{a}(b,h),V_n^{wa}(b,h)\},
\]

where $V_n^{a}(b,h)$ and $V_n^{wa}(b,h)$ are value functions corresponding to the agent makes adjustment on her housing asset and that without adjustment, respectively, given by

\[
V_n^{d}(b,h) = \max_{\{c, s, b',h'\}}  \frac{(c^\alpha s^{1-\alpha})^{1-\gamma}}{1-\gamma} + \beta V_{n+1}(b',h')
\]

s.t.
\begin{align*}
c+ R(t)s+\frac{b'}{(1+r_{t})}+P(t)(h'-h)&= y_n+b+R(t)h' \\
%b'&\ge -M_h h'P'I_{h'\ge\underbar{h}} -M_y \sum_{k=n+1}^{N}\frac{y_k}{\Pi_{l=n+1}^{k}(1+r_{t+l-n-1})}\\
b'&\ge \max\{-M_h h'P'I_{h'\ge\underbar{h}}, -M_y \sum_{k=n+1}^{N}\frac{y_k}{\Pi_{l=n+1}^{k}(1+r_{t+l-n-1})}\}\\
h'&\in \left\{ \begin{array}{ll}
\{[0,\max \{h-\underbar{h},0\}],[h+\underbar{h},+\infty)\} & \mbox{if $d=a$}\\
\{h\}& \mbox{otherwise}
\end{array}
\right.
\end{align*}

for $d\in\{a,aw\}$.

Before laying out the definition of an equilibrium, we first summarize the optimal decisions by the age $n$ agent as the following policy functions,
\[
b'_n=p_n^b(b,h),\quad h'_n=p_n^h(b,h)
\]
by solving the above defined problem.
\vspace{2cm}

\paragraph{Equilibrium.}
The steady state equilibrium in this economy is defined as: a set of individual value and policy functions $\{V_n,b'_n,h'_n\}_{n=1}^{N}$, a set of prices $r_t,R(t),P(t)$, such that

\begin{enumerate}
  \item $\{V_n,p_n^b,p_n^h\}_{n=1}^{N}$ solve the households' optimization problems.
  \item $r_t,R(t),P(t)$ are constant.
  \item All the spot markets clear at all times.
  \begin{enumerate}[(i)]
    \item  $\sum_{n=1}^{N} b'_n = 0$
    \item  $\sum_{n=1}^{N} (h'_n-h_n)= 0$, which comes for free, since we have
    \[
    \sum_{n=1}^{N} (h'_n-h_n)= \sum_{n=1}^{N-1} h_{n+1} + h'_N-\sum_{n=2}^{N} h_{n}-h_1 = h_N-h_1=0
    \]
    \item  $\sum_{n=1}^{N} s_n = H$
  \end{enumerate}
  \item Cross-sectional housing asset holding position generate $H$, i.e., $\sum_{n=1}^{N} h_n= H$.
\end{enumerate}


\iffudongxingtongonly
NTGB: What is the relationship between $P$ and $R$? Relatedly, how can we guarantee no one will hold a house level $0<h<\underbar{h}$ ever?
\fi

\pagebreak



\subsection*{Solution}

Since agents die with certainty at age $N$, we can solve for $(V_N(b,h),b'_N,h'_N)$ first and do backward induction to find all the other value and policy functions $(V_k(b,h),b'_k,h'_k)$, $\forall k<N$.


\subsubsection*{Mathematics}

\begin{enumerate}
  \item First of all, it's easy to solve for $(V_N(b,h),b'_N,h'_N)$. Since agents just need to optimally allocate all of his resources across $c$ and $s$.

  FOCs w.r.t. $c$ and $s$ give us
\[
\frac{s}{c}= \frac{1-\alpha}{\alpha R}.
\]
Thus we have
\[
\frac{1}{\alpha}c= y_N+(1+r)b+Rh+PhI_{h\ge\underbar{h}}.
\]
Then we can get
\[
V_N(b,h) = \mbox{log}([(1+r_t)b+y_N+(R(t)+I_{h\ge\underbar{h}} P(t))h])+\alpha\mbox{log}(\alpha)+(1-\alpha)\mbox{log}(\frac{(1-\alpha)}{R(t)}),
\]
 where $I_{h\ge\underbar{h}}$ is the characterization function.
  \item Solve for $V_n^{wa}(b,h)$.

\[
V_n^{wa}(b,h) = \max_{\{c, s, b'\}}  \frac{(c^\alpha s^{1-\alpha})^{1-\gamma}}{1-\gamma} + \beta V_{n+1}(b',h)
\]

FOC w.r.t. $b'$ gives us

\[
[\frac{1}{y_n+(1+r)b+(R+P)h-b'} - \beta \frac{\partial V_{n+1}(b',h)}{\partial b'}][b'-\underbar{b}'_n(h)]=0
\]


  \item Solve for $V_n^{a}(b,h)$.

  \[
V_n^{a}(b,h) = \max_{\{c, s, b',h'\}}  \frac{(c^\alpha s^{1-\alpha})^{1-\gamma}}{1-\gamma} + \beta V_{n+1}(b',h')
\]

FOCs w.r.t. $b'$ and $h'$ give us
\[
[\frac{1}{y_n+(1+r)b+(R+P)h-b'-Ph'} - \beta \frac{\partial V_{n+1}(b',h')}{\partial b'}][b'-\underbar{b}'_n(h')]=0
\]
and
\[
[\frac{1}{y_n+(1+r)b+(R+P)h-b'-Ph'} - \beta \frac{\partial V_{n+1}(b',h')}{\partial h'}][h'-(h+\underbar{h})][h'-(h-\underbar{h})]=0
\]

\item Then we do backward induction to get $(V_k(b,h),b'_k,h'_k)$, $\forall k<N$.

Due to the presence of boundary conditions, we cannot find analytical solutions. We will solve the problem numerically.

\end{enumerate}


\iffudongxingtongonly

\subsubsection*{Numerical implementation}

\noindent \emph{Step 0:} Start from an initial guess of prices $r$, $R$ and $P$.
\vspace{2 mm}

\noindent \emph{Step 1:} Find the analytical value function at age $N$.
\[
V_N(b,h) = \mbox{log}([(1+r_t)b+y_N+(R(t)+I_{h\ge\underbar{h}} P(t))h])+\alpha\mbox{log}(\alpha)+(1-\alpha)\mbox{log}(\frac{(1-\alpha)}{R(t)}),
\]
Select the interpolation grid for $(b,h)$ used in the approximation of both household's continuation value functions and policy functions (we might want to alow for different grids across ages to get more accurate results).
\vspace{2 mm}


\noindent \emph{Step 2:} Set up the interpolated continuation value function at age $N$. Solve for the value function and policy function at age $n\le N$ by backward induction. The following is detailed sub-steps for one step calculation.
\begin{enumerate}
% \item Solve for $V_{N-1}^{wa}(b,h)$ and $b^{'wa}_{N-1},h^{'wa}_{N-1}$. For any $(b,h)$ grid point, solve the constrained one dimensional optimization problem defined earlier, where $b'\in[-M_h hPI_{h\ge\underbar{h}}-\frac{M_y y_N}{1+r}, y_{N-1}+(1+r)b+Rh]$. Note that the upper bound of $b'$ comes from the positive consumption requirement.\footnote{For $(b,h)$ grid such that there exists no $b'$ that can generate positive consumption, we set $V_{N-1}^{wa}(b,h)$ to be a negative enough number. In this case, we also assign $b^{'wa}_{N-1}=0$, which does not matter in any case, since rational agents will never reach such a state.}

  \item Solve for $V_{N-1}^{wa}(b,h)$ and $b^{'wa}_{N-1},h^{'wa}_{N-1}$. For any $(b,h)$ grid point, solve the constrained one dimensional optimization problem defined earlier, where $b'\in[\max \{-M_h hPI_{h\ge\underbar{h}}, -\frac{M_y y_N}{1+r}\}, y_{N-1}+(1+r)b+Rh]$. Note that the upper bound of $b'$ comes from the positive consumption requirement.\footnote{For $(b,h)$ grid such that there exists no $b'$ that can generate positive consumption, we set $V_{N-1}^{wa}(b,h)$ to be a negative enough number. In this case, we also assign $b^{'wa}_{N-1}=0$, which does not matter in any case, since rational agents will never reach such a state.}

 % \item Solve for $V_{N-1}^{a}(b,h)$ and $b^{'a}_{N-1},h^{'a}_{N-1}$. For any $(b,h)$ grid point, solve the constrained two dimensional optimization problem defined earlier. We do the optimization in two steps. Firstly, for any given $h'$, we solve the the constrained one dimensional optimization problem defined similarly as sub-step 1. We then search along possible $h'$ to find the optimal $h^{'a}_{N-1}$. Note that $h'\in\{[0,\max \{h-\underbar{h},0\}],[h+\underbar{h},\min \{ \frac{1}{(1-M_h)P} (\frac{M_y y_N}{1+r}+y_{N-1}+(1+r)b+(R+P)h),h_{max}\}]\}$, again from the positive consumption requirement.\footnote{Similarly, for $(b,h)$ grid such that there exists no $h'$ that can generate positive consumption, we set $V_{N-1}^{a}(b,h)$ to be a negative enough number. In this case, we also assign $h^{'a}_{N-1}=0$.}
  \item Solve for $V_{N-1}^{a}(b,h)$ and $b^{'a}_{N-1},h^{'a}_{N-1}$. For any $(b,h)$ grid point, solve the constrained two dimensional optimization problem defined earlier. We do the optimization in two steps. Firstly, for any given $h'$, we solve the the constrained one dimensional optimization problem defined similarly as sub-step 1. We then search along possible $h'$ to find the optimal $h^{'a}_{N-1}$. Note that $h'\in\{[0,\max \{h-\underbar{h},0\}],[h+\underbar{h},\min \{ \min \{\frac{1}{P} (\frac{M_y y_N}{1+r}+y_{N-1}+(1+r)b+(R+P)h),\frac{1}{(P-M_h} (y_{N-1}+(1+r)b+(R+P)h)\},h_{max}\}]\}$, again from the positive consumption requirement.\footnote{Similarly, for $(b,h)$ grid such that there exists no $h'$ that can generate positive consumption, we set $V_{N-1}^{a}(b,h)$ to be a negative enough number. In this case, we also assign $h^{'a}_{N-1}=0$.}

  \item Get $V_{N-1}(b,h)=\max \{V_{N-1}^{a}(b,h),V_{N-1}^{wa}(b,h)\}$ and corresponding $b'_{N-1},h'_{N-1}$.
\end{enumerate}
\vspace{2 mm}

\noindent \emph{Step 3:} Use the initial endowment of the new born agent and those policy functions from step 2 to get optimal decisions of all the agents in the economy. Stop if all markets clear. Otherwise, update prices and go back to step 1.


\fi

\pagebreak

\subsection*{Results}

We solve the model with the following parameter values.

\begin{table} [h!]

\caption{Summary of Parameters}
\footnotesize
\begin{adjustwidth}{-1cm}{-1cm}%
\centering
\begin{tabular}{l l l l}
\hline
& Parameter & Value & Target/source\\
\hline
Relative risk aversion & $\gamma$ & 1.00 &  Standard value \\
Utility weight on consumption & $\alpha$ & 0.85 & ratio of the housing stock to output \\
Utility weight on consumption & $\alpha$ & 0.85 & ratio of the housing services to output, 12\% \\
Life span & $N$ & 60 &  \\
Endowment when working &$y_{n\le40}$ &  &   \\
Endowment after retirement & $y_{n>40}$ &  &  40\% of average working income \\
Discount rate & $beta$ & 0.95 &   \\
Max debt, fraction of house &$M_h$ & 0.75 &   \\
Max debt, fraction future income & $M_y$ &0.25 & \\
Minimum housing transaction size &    $\underbar{h}$    &  1   & housing value over current income ratio\\
Housing stock &  $H$   &    & ratio of the housing stock to output\\

\hline
\end{tabular}
\end{adjustwidth}
\label{table_calibration}
\end{table}

\vspace{1cm}
We first solve the model with integer restriction on housing consumption. With the above parameter setting, we get the following market clearing prices

\[r = 0.2484, \quad R=0.0428, \quad P=5.0021, \quad \frac{R}{P}=0.0497.\]

\vspace{1cm}

The following graphs depict the cross-sectional distributions of bond holding, housing asset position, as well as the consumption profile.

\vspace{2cm}

\begin{figure}[h!]
\begin{center}
		\includegraphics[height=2.5in,width=4in]{iipath.eps}
		\caption{Income process}
\end{center}\end{figure}

\begin{figure}[!ht]
\begin{center}
		\includegraphics[height=2.5in,width=4in]{bbpath.eps}
		\caption{Bond holding}
\end{center}\end{figure}

\begin{figure}[!ht]
\begin{center}
		\includegraphics[height=2.5in,width=4in]{hhpath.eps}
		\caption{Housing asset position}
\end{center}\end{figure}

\begin{figure}[!ht]
\begin{center}
		\includegraphics[height=2.5in,width=4in]{ccpath.eps}
		\caption{Consumption profile}
\end{center}\end{figure}




\end{document}
